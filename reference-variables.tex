% !TEX root = /Users/cmerlo/cs1textbook/cs1textbook.tex

\chapter{Reference Variables}
\label{chapter:reference-variables}

\minitoc

\section{Built-In Data Types}

The Java language provides a few ways to represent data.  In Chapter \ref{chapter:info-numbers} we learned a couple of the ways Java allows us to store numbers.  Truth values were discussed in Chapter \ref{chapter:if}.  Then, in Chapter \ref{chapter:char}, we learned how to store characters and strings.  But, as they said on Sesame Street, one of these things is not like the others.

\subsection{Primitive Data Types}

In Section \ref{section:string}, it was noted that the data types \texttt{int}, \texttt{double}, \texttt{boolean}, and \texttt{char} are \textbf{primitive data types}\index{Primitive Data Types}, without discussing what that means.  For now, we will define a variable of a primitive data type as one that stores a value directly, but this won't mean much until we explore the other kind of data type.

\subsection{Classes Are Also Data Types}
We also learned in \ref{section:string} that the \texttt{String} data type is part of the Java language.  \texttt{String} is also the name of a Java class.  A \textbf{class} serves as a data type in Java, and in this chapter we will explore how to create our own data types.

Java programmers use the word \textit{class} to describe a group of objects with similar characterisics or uses; for instance, acoustic 12-string guitars are one class of musical instruments, and wide receivers are one class of football players.  When we define a class, we specify what objects of that class will be like, and we explain what sort of tasks these objects will be able to perform.

\begin{defn}{Class}
    A \textbf{class}\index{Class!Definition} is a Java data type.  We programmers can create our own data types by defining our own classes.  A typical Java class defines the \textit{properties} and \textit{behaviors} of the variables of this type that will be created.
\end{defn}

\section{Primitive Variables and Reference Variables}

Before starting to discuss how to create our own Java classes, we should examine what happens in memory when we create variables of those classes.  In Section \ref{section:string} we learned that a variable of type \texttt{String} must be \textit{instantiated} using code like:

\begin{minted}{java}
String s1 = new String( "Hello world!" );
\end{minted}

The way this \texttt{String} is stored, however, is quite different from how an integer is stored.  The value is not directly stored in the variable.  Consider this code:

\begin{minted}{java}
int x1 = 15;
String s1 = new String( "Hello world!" );
\end{minted}

Figure \ref{fig:reference-variable} shows how the values are stored.

\begin{figure}[ht]
    \begin{center}
        \sffamily
        \begin{subfigure}{0.2\textwidth}
            \begin{center}
                \begin{tikzpicture}[var/.style={draw=nccblue,fill=nccorange,text=white,minimum width = width{"Hello world!"}, minimum height=1.25pt,inner sep=2pt,outer sep=2pt}]
                    \node (x1) [var] {15};
                    \node [above = 2mm of x1] {x1};
                \end{tikzpicture}
        \end{center}
            \caption{\mintinline{java}{int x1 = 15;}}
        \end{subfigure}%
        \begin{subfigure}{0.4\textwidth}
            \begin{center}
                \begin{tikzpicture}[var/.style={draw=nccblue,fill=nccorange,text=white,minimum height=2pt,inner sep=2pt,outer sep=2pt},data/.style={draw=nccblue,fill=white,text=nccblue,inner sep=2pt,outer sep=2pt}]
                    \node (s1) [var] {\textcolor{nccorange}{X}\textcolor{nccblue}{\raisebox{0.25ex}{$\bullet$}}\textcolor{nccorange}{X}};
                    \node (hw) [data,right = 1cm of s1] {Hello world!};
                    \node [above = 2mm of s1] {s1};

                    \path [->,draw,thick,nccblue] (s1.center) -- (hw);
                \end{tikzpicture}
            \end{center}
            \caption{\mintinline{java}{String s1 = new String( "Hello world!" );}}
        \end{subfigure}
    \end{center}
    \caption{How Integers and Strings Are Stored in Memory}
    \label{fig:reference-variable}
\end{figure}

The actual string is stored some place in memory away from the variable \texttt{s1}, and what \texttt{s1} stores is the numeric location of this place in memory.  We say that \texttt{s1} is a \textbf{reference variable} because it contains a \textit{reference} to the string ``Hello world!''.

\begin{defn}{Reference Variable}
    A \textbf{reference variable}\index{Reference Variable} is a variable that stores a \textit{reference} to, or the memory address of, a piece of data stored somewhere else in memory.
\end{defn}

Throughout this Part of the book, we will be developing an application as a case study.  Specifically, we will be designing a program that children can use to draw geometric shapes on the screen, and learn about some of these shapes' important characteristics like area, perimeter, volume, etc.
