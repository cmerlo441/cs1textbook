% !TEX root = /Users/cmerlo/cs1textbook/cs1textbook.tex

\chapter{Reference Variables}
\label{chapter:reference-variables}

In Section \ref{section:string} we learned that a variable of type \texttt{String} must be \textit{instantiated} using code like:

\begin{minted}{java}
String s1 = new String( "Hello world!" );
\end{minted}

The way this \texttt{String} is stored, however, is quite different from how an integer is stored.  The value is not directly stored in the variable.  Consider this code:

\begin{minted}{java}
int x1 = 15;
String s1 = new String( "Hello world!" );
\end{minted}

Figure \ref{fig:reference-variable} shows how the values are stored.

\begin{figure}[ht]
    \begin{center}
        \sffamily
        \begin{subfigure}{0.2\textwidth}
            \begin{center}
                \begin{tikzpicture}[var/.style={draw=nccblue,fill=nccorange,text=white,minimum width = width{"Hello world!"}, minimum height=1.25pt,inner sep=2pt,outer sep=2pt}]
                    \node (x1) [var] {15};
                    \node [above = 2mm of x1] {x1};
                \end{tikzpicture}
        \end{center}
            \caption{\mintinline{java}{int x1 = 15;}}
        \end{subfigure}%
        \begin{subfigure}{0.4\textwidth}
            \begin{center}
                \begin{tikzpicture}[var/.style={draw=nccblue,fill=nccorange,text=white,minimum height=2pt,inner sep=2pt,outer sep=2pt},data/.style={draw=nccblue,fill=white,text=nccblue,inner sep=2pt,outer sep=2pt}]
                    \node (s1) [var] {\textcolor{nccorange}{X}\textcolor{nccblue}{\raisebox{0.25ex}{$\bullet$}}\textcolor{nccorange}{X}};
                    \node (hw) [data,right = 1cm of s1] {Hello world!};
                    \node [above = 2mm of s1] {s1};

                    \path [->,draw,thick,nccblue] (s1.center) -- (hw);
                \end{tikzpicture}
            \end{center}
            \caption{\mintinline{java}{String s1 = new String( "Hello world!" );}}
        \end{subfigure}
    \end{center}
    \caption{How Integers and Strings Are Stored in Memory}
    \label{fig:reference-variable}
\end{figure}

The actual string is stored some place in memory away from the variable \texttt{s1}, and what \texttt{s1} stores is the numeric location of this place in memory.  We say that \texttt{s1} is a \textbf{reference variable} because it contains a \textit{reference} to the string ``Hello world!''.

\begin{defn}{Reference Variable}
    A \textbf{reference variable}\index{Reference Variable} is a variable that stores a \textit{reference} to, or the memory address of, a piece of data stored somewhere else in memory.
\end{defn}
