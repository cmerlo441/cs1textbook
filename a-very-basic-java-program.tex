% !TEX root = /Users/cmerlo/cs1textbook/cs1textbook.tex

\chapter{A Very Basic Java Program}

\minitoc

The discussions in this chapter are meant to acquaint you with the layout and structure of a typical Java program.  You are \textbf{\textit{not expected}} to gain a great deal of understanding here; this is just to lay out the scaffolding of Java.  You will come to understand the details as you work through the textbook.

\section{Basic Program Components}

\subsection{Classes}

We will defer a good working definition of the word \textit{class} as it's used in Java until later.  For now, understand that 99\% of all Java program code exists within the definition of a class.  The basic layout of a class is:

\begin{javaformat}{Basic Class Layout}
\begin{minted}{java}
public class <class-name> {
    <declarations>

    <method-definitions>
}
\end{minted}
\end{javaformat}

\subsection{Methods}

As with classes, a good working definition of what a method is awaits you later in this textbook (Section \ref{section:method-calls} to be precise).  For now, just understand that the most basic Java program requires one method, and it must be called \lstinline{main}.  The \textit{main method} must conform to this layout:

\begin{javaformat}{Main Method}
\begin{minted}{java}
    public static void main( String args[ ] ) {
        <declarations>

        <statements>
    }
\end{minted}
\end{javaformat}

\subsection{Statements}
A useful Java program must contain some \textit{statements}.  A fuller explanation about statements starts in Chapter \ref{chapter:info-numbers}, but the magic code you see in Section \ref{sec:helloworld} contains one \textit{declaration statement}, one \textit{assignment statement}, and one \textit{output statement}.

\section{Hello World!}
\label{sec:helloworld}

The grand tradition of programming language textbooks is to introduce the language with a program that makes the phrase ``Hello world!''\ appear on the screen.  This author is a fan of that tradition.

Note that this is a listing for a complete program; this source code can be compiled and run as-is, as long as it's stored in a source code file called \texttt{HelloWorld.java}.  Instructions for how to do this appear in Appendix \ref{appendix:compiling}.

Also note that this is the textbook's first use of \textit{magic code}.  You are not meant to understand magic code right away -- while some of it might make sense now, its meaning will become clear later.

\begin{magic}{Hello World}
\begin{minted}{java}
public class HelloWorld {
    public static void main( String args[] ) {
        String message;

        message = new String( "Hello world!" );
        System.out.println( message );
    }
}
\end{minted}
\end{magic}
