% !TEX root = cs1textbook.tex

\chapter{ASCII}
\label{appendix:ascii}

\minitoc

\textbf{ASCII}\index{ASCII}, the American Standard Code for Information Interchange, is a table that relates \textit{characters}, which were discussed in Section \ref{section:characters}, with numbers.  It was devised in the 1960s by the organization now called ANSI, the American National Standards Institute.  By assigning a numeric code to each character that a computer might represent, it became possible for sequences of these characters to be transmitted from one computer to another.  Data are sent across networks in numeric form.

In this chart, the high byte is listed in the first column, and the low byte is listed in the first row.  To determine the ASCII value of the character `E', for instance, look left to its high byte 4, and then up to its low byte 5, to find the hexadecimal value 45, equivalent to decimal 69.

Note that ASCII is technically a seven-bit encoding scheme, and so officially there are no ASCII values higher than 7F$_{16}$, or $127_{10}$.  However, since PCs use eight bits to store a character, several operating systems define their own encodings for ASCII values between 80$_{16}$ ($128_{10}$) and FF$_{16}$ ($255_{10}$).

%\begin{landscape}
\begin{center}
\footnotesize
\bgroup
%\def\arraystretch{1.5}
\begin{tabular}{r | *{16}{c}}
 & 0 & 1 & 2 & 3 & 4 & 5 & 6 & 7 & 8 & 9 & A & B & C & D & E & F\\
\hline\\
0 & NUL & SOH & STX & ETX & EOT & ENQ & ACK & BEL & BS & HT & LF & VT & FF & CR & SO & SI\\

1 & DLE & DC1 & DC2 & DC3 & DC4 & NAK & SYN & ETB & CAN & EM & SUB & ESC & FS & GS & RS & US\\

2 & Space & ! & \verb-"- & \verb-#- & \verb-$- & \verb-%- & \verb-&- & \verb-'- & \verb-(- & \verb-)- & \verb-*- & \verb-+- & \verb-,- & \verb|-| & \verb-.- & \verb-/-\\

3 & \verb-0- & \verb-1- & \verb-2- & \verb-3- & \verb-4- & \verb-5- & \verb-6- & \verb-7- & \verb-8- & \verb-9- & \verb|:| & \verb|;| & \verb|<| & \verb|=| & \verb->- & \verb-?-\\

4 & \verb-@- & \verb-A- & \verb-B- & \verb-C- & \verb-D- & \verb-E- & \verb-F- & \verb-G- & \verb-H- & \verb-I- & \verb-J- & \verb-K- & \verb-L- & \verb-M- & \verb-N- & \verb-O-\\

5 & \verb-P- & \verb-Q- & \verb-R- & \verb-S- & \verb-T- & \verb-U- & \verb-V- & \verb-W- & \verb-X- & \verb-Y- & \verb-Z- & \verb-[- & \verb-\- & \verb-]- & \verb-^- & \verb-_-\\

6 & \verb-`- & \verb-a- & \verb-b- & \verb-c- & \verb-d- & \verb-e- & \verb-f- & \verb-g- & \verb-h- & \verb-i- & \verb-j- & \verb-k- & \verb-l- & \verb-m- & \verb-n- & \verb-o-\\

7 & \verb-p- & \verb-q- & \verb-r- & \verb-s- & \verb-t- & \verb-u- & \verb-v- & \verb-w- & \verb-x- & \verb-y- & \verb-z- & \verb-{- & \verb-|- & \verb-}- & \verb-~- & DEL\\
\end{tabular}
\egroup
\end{center}
%\end{landscape}

% \hline
% 10000 & 16  & 0010 & DLE & Data link escape\\
% 10001 & 17  & 0011 & DC1 & Device control 1\\
% 10010 & 18  & 0012 & DC1 & Device control 2\\
% 10011 & 19  & 0013 & DC1 & Device control 3\\
% 10100 & 20  & 0014 & DC1 & Device control 4\\
% 10101 & 21  & 0015 & NAK & Negative acknowledgment\\
% 10110 & 22  & 0016 & SYN & Synchronous Idle\\
% 10111 & 23  & 0017 & ETB & End of transmission block\\
%
% \hline
% \end{tabular}
