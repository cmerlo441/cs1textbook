% !TEX root = /Users/cmerlo/cs1textbook/cs1textbook.tex

\chapter{Characters, Strings, and Method Calls}
\label{chapter:char}

\minitoc

\section{Characters}
\label{section:characters}

A \textit{character}, generally speaking, is the sort of thing you get one of by pressing and releasing one key on your keyboard.  A letter, a digit, and a piece of punctuation are all examples of a character.

In Java, we use the datatype \mintinline{java}{char}\index{Primitive Data Types!char@\mintinline{java}{char}} to represent one character.  A \textit{character literal} must be surrounded by single quotes.  Consider the following examples:

\begin{minted}{java}
char c1 = 'W';
char c2 = '?';
char c3 = ' ';  // a space
char c4 = '9';
\end{minted}

\mintinline{java}{char} values can be compared; for example, \mintinline{java}{'b'} is greater than \mintinline{java}{'a'}.

Note that only a single character can appear within the single quotes, so a character literal like \mintinline{java}{'abc'} is illegal.  However, certain \index{Digraph}digraphs exist to allow us to represent \textit{special characters}.  These digraphs are frequently referred to as \index{Escape sequence}\textit{escape sequences}.  There is no way, for instance, to store the single quote character in a \mintinline{java}{char} variable without using a special character.  Typing \mintinline{java}{'''} won't do it; rather than treating the third quote as a match to the first, the computer will consider the second one a match to the first, making the third one an unmatched opening quote.

The proper method to represent the single quote character is to \textit{escape} it with a backslash:

\begin{minted}{java}
char single_quote = '\'';
\end{minted}

The backslash escapes the second single quote, changing its meaning from ``close that opening quote'' to ``this is just a character''.  The backslash and the escaped quote together make one special character.  Another special character that you will use frequently is the newline character, \mintinline{java}{'\n'}.

\section{Strings and Objects}
\label{section:string}

A \mintinline{java}{String}\index{String@\mintinline{java}{String}} is a sequence of characters.  Instead of using single quotes like we do to specify a character literal, we specify a string literal with double quotes, like \mintinline{java}{"This is a string"}.

The data types that we have discussed so far -- \mintinline{java}{int}, \mintinline{java}{double}, \mintinline{java}{boolean}, and \mintinline{java}{char} -- are examples of \textbf{primitive data types}\index{Primitive Data Types}.  A variable of a primitive data type stores one data value and nothing else.  Strings, however, are stored using a different kind of variable.  Consider these statements:

\begin{minted}{java}
String s1;
s1 = new String( "Hello world!" );
\end{minted}

Because a string is a special kind of variable called an \textit{object}, we create one using that keyword \mintinline{java}{new}.  Programmers will refer to this process as \textit{instantiation}.  Also, because a string is an object, we can send it messages, ask it questions, and tell it to perform tasks.  Objects will be discussed in depth in a later chapter.

\begin{defn}{Instantiation}\label{defn:instantiation}
The act of creating an object is called \textbf{instantiation}\index{Instantiation}.  Programmers say things like ``I \textit{instantiated} a \mintinline{java}{String} object'' or ``I created an \textit{instance} of the \mintinline{java}{String} class''.
\end{defn}

\subsection{Length}
%\newpagecolor{yellow}\afterpage{\restorepagecolor}
One of the questions we can ask a \mintinline{java}{String} is how long it is.  Notice that the string ``Hello world!'' consists of twelve characters -- ten letters, a space, and an exclamation point.  Therefore, the following code will make the number 12 be stored in the variable \mintinline{java}{characterCount}:

\begin{minted}{java}
String s1;
int characterCount;

s1 = new String( "Hello world!" );
characterCount = s1.length();
\end{minted}

We can create two different \mintinline{java}{String} objects, ask each one its length in the same exact way, and receive the correct answer from each one.

\begin{minted}{java}
String s1 = new String( "abcdef" );
String s2 = new String( "ghij" );
int count1;
int count2;

count1 = s1.length();  // count1 stores 6
count2 = s2.length();  // count2 stores 4
\end{minted}

We could even use a simple if-else statement from Section \ref{section:ifelse} to store only the length of the longer String:

%\lstinputlisting[label=code:longerstring,caption=Determine the Longer String]{code/LongerString.java}

\begin{listing}[H]
\caption{Determine the Longer String}
\inputminted{java}{code/LongerString.java}
\label{code:longerstring}
\end{listing}

Notice that the two values being compared on line 9 are both returned from method calls, as described next.

\section{Method Calls}
\label{section:method-calls}
The tool called \mintinline{java}{length()} from the previous section is an example of a \textit{method}.

\begin{defn}{Method}
A \textbf{method} is a collection of declarations and statements that performs a task.  A method must be \textit{called}, or \textit{invoked}, in order to run.
\end{defn}

Because the \mintinline{java}{length()} method is part of the \mintinline{java}{String} class, we can ask for the length of every \mintinline{java}{String} object we make, like \mintinline{java}{s1} and \mintinline{java}{s2} in the previous section.

Further, because the \mintinline{java}{length()} method is part of the \mintinline{java}{String} class, every call to the method must be preceded by the name of the \mintinline{java}{String} object whose method is being called.  The code \mintinline{java}{s1.length()} calls \mintinline{java}{s1}'s length method, and the code \mintinline{java}{s2.length()} calls \mintinline{java}{s2}'s length method.  Each of these calls runs the same code, but by referring to two different \mintinline{java}{String} objects, we get two different lengths.

Notice that a statement like \mintinline{java}{int c = length();} is against the rules, because that statement doesn't specify whose length to calculate.

\subsection{Return Values and Return Types}
A method like \mintinline{java}{length()} from the \mintinline{java}{String} class is said to \textit{return} a value to its caller.  Specifically, it returns an \mintinline{java}{int} to its caller.  That's why a statement like \mintinline{java}{count1 = s1.length()} makes sense to the computer, because the lvalue and the rvalue are of the same type.

To avoid a type mismatch error (recall Definition \ref{defn:typemismatch}), you must know the method's \textit{return type}, and make sure the lvalue has the same type.  A statement like \mintinline{java}{String s3 = s1.length();} would not be turned into a program, because the \mintinline{java}{length()} method does not return a \mintinline{java}{String}.

\subsection{Parameters}
Another question we can ask of a \mintinline{java}{String} is whether it contains a particular character.  This code asks a \mintinline{java}{String} whether it contains a letter H:

\begin{minted}{java}
String message = new String( "Hello World!" );
int position = message.indexOf( 'H' );
\end{minted}

The \mintinline{java}{indexOf} method returns the \textit{index}, or numerical position, of the letter within the string.  This question wouldn't make any sense if we didn't specify \textit{which} letter we're interested in locating.  The \mintinline{java}{'H'} within the parentheses of that method call is required, so that the \mintinline{java}{indexOf} method knows what to look for.  This information -- the letter H -- is called a \textit{parameter}.

\begin{defn}{Parameter}
A \textbf{parameter}\index{Parameter} is a piece of information sent to a method to specify how the method should work.
\end{defn}
