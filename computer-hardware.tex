%!TEX root = /Users/cmerlo/cs1textbook/cs1textbook.tex

\chapter{Computer Hardware}

\minitoc

There are lots of gadgets in the world these days that can be described as computers, from what we typically think of as ``computers'' -- metal and plastic boxes with standalone monitors and keyboards attached -- to small form factor machines like the Mac Mini, to video game consoles like the Xbox, to tablets and mobile phones, to tiny system-on-a-chip designs like the Raspberry Pi.  All of these devices have a few things in common, though.

This chapter will not make you an expert in computer hardware, but you will learn some important things about how the programs you write will interact with the computer's components.

\section{The Central Processing Unit}

Every computer has a \index{Central Processing Unit}\textbf{Central Processing Unit} (CPU), which is responsible for actually performing most of the instructions in any computer program.  It's cool.

\section{Memory}

...  As you will see in Section \ref{sec:using-memory}, ...

\section{Secondary Storage}

\section{Input/Output Devices}

\subsection{The Console}
\label{subsection:console}

When computers were first being developed, before the days of Bluetooth trackpads and touch screens, the only way to enter information into the computer efficiently was with a keyboard, and the only way to see what the computer was doing efficiently was on a screen.\footnote{Actually, before all that, computers could only do input and output with punched cardboard cards!}  These two devices were often referred to collectively as the console.

\index{Console}
\begin{defn}{Console}
A computer's \textbf{console}, in hardware terms, is its keyboard and monitor.  In software terms, the word console is used as a metonym for non-graphical input and output, also called \textit{character-based} input and output.
\end{defn}

\subsection{Graphical User Interfaces}
