% !TEX root = /Users/cmerlo/cs1textbook/cs1textbook.tex

\chapter{Introduction to Creating Classes}
\label{chapter:classes}

\minitoc

%\section{The \texttt{Point} Class}

We will begin our geometry case study by creating a data type to represent the smallest unit of geometry -- the point -- and build up in complexity from there.  \href{https://en.wiktionary.org/wiki/point}{Wiktionary defines a geometrical point} as ``A zero-dimensional mathematical object representing a location in one or more dimensions; something considered to have position but no magnitude or direction''.  We will define our \mintinline{java}{Point} class to represent an object on a two-dimensional Cartesian plane, since this is a good analogy for a computer screen.  A detail that will become important later is that several computer graphics paradigms, including the one we'll be working under, fix the origin of the screen's Cartesian plane at the top left of the screen, so that $x$ values increase from right to left and $y$ values increase from the top of the screen to the bottom; this results in all $x$ and $y$ values being non-negative.  A monitor with 1024$\times$768 resolution, therefore, would find the point $(1023,767)$ at the bottom right of the screen, as in Figure \ref{fig:monitor}.

\begin{figure}[ht]
    \color{nccblue}
    \begin{center}
        \begin{tikzpicture}
            \draw [thick,nccblue,fill=nccorange] (0,0) -- (5,0) -- (5,4) -- (0,4) -- (0,0) {};
            \draw [thin,nccblue] (1,0) -- (0.5,-0.5) -- (4.5,-0.5) -- (4,0);

            \draw (5,0) node [circle,fill=nccblue] {};
            \draw (0,4) node [circle,fill=nccblue] {};

            \draw (-0.5,3.65) node {$(0,0)$};
            \draw (6,0.3) node {$(1023,767)$};
        \end{tikzpicture}
    \end{center}
    \caption{Cartesian Coordinates of a Computer Monitor with 1024$\times$768 Resolution}
    \label{fig:monitor}
\end{figure}

\section{Class Header}

Every class definition begins with a class \textit{header}.  The header defines the class' visibility and its name.  This is how a simple class header looks:

\begin{javaformat}{Class Header}
\begin{minted}{text}
<visibility-modifier> class <identifier>
\end{minted}
\end{javaformat}

The visibility modifier for classes will be \mintinline{java}{public} throughout this textbook.

In Section \ref{sec:using-memory} we learned some rules that Java imposes on us regarding what we can choose as an identifier:
\bi
\item An identifier \textbf{must} start with a letter.
\item An identifier \textbf{must} only contain letters, digits, and the underscore character (\lstinline{'_'}).
\item An identifier \textbf{must not} be the same as a \textit{reserved word} in Java (like \lstinline{class} or \lstinline{static}).
\ei

Section \ref{sec:using-memory} also lists some guidelines regarding naming a variable.  The guidelines for naming a class are a bit different, however:
\bi
\item The name of a class should \textit{start with a capital letter}.
\item Classes define \textit{things}, so the name should be a \textit{noun}.
\ei

Finally, each class should have a JavaDoc comment above the header that explains what this class provides to the programmer.

For the \texttt{Point} class, the header will look like the first line of this code:

\begin{minted}{java}
/**
 * Provides a two-dimensional point on a Cartesian plane.
 * <p>
 * All x and y values are non-negative.
 */
public class Point {  // This line is the header
    /* The body of the class will go here */
}
\end{minted}

Notice that I have put the opening curly brace of the body at the end of the header.  Some programmers choose instead to place an opening curly brace on the next line.

\section{Instance Variables}

After the class header, the rest of the class' definition -- the \textit{body} of the class -- appear inside curly braces.  The first thing that most programmers put at the top of the class' body is a list of the class' \textit{properties}.  This list consists of variable declarations that describe the data an object of this class will contain.

One instance of the \texttt{Point} class should contain enough information to complete an ordered pair in $(x,y)$ format.  We'll assume some knowledge about computer monitors -- specifically, that there are no partial pixels -- and specify that both the $x$ value and the $y$ value must be integers.

\begin{minted}{java}
public class Point {
    int x;  // x component of Cartesian ordered pair
    int y;  // y component of Cartesian ordered pair
}
\end{minted}

Variables like these, declared in class scope, are called \textbf{instance variables}\index{Variable!Instance variable}, because every instance of the \texttt{Point} class that we create will store values in these variables.

In a different class, we will create a \texttt{Point} object and place some data in these instance variables:

\begin{minted}{java}
public class TestPoint {
    public static void main( String args[] ) {
        Point p1 = new Point();  // p1 represents a point on the screen
        p1.x = 10;   // p1's x coordinate is now 10
        p1.y = 15;   // p1's y coordinate is now 15

        System.out.println( "The point is at (" + p1.x + "x" + p1.y + ")." );
    }
}
\end{minted}

So far, it seems like we've solved our problem.  The output from this program will be what you expect: \texttt{The point is at (10x15).}  There exists, however, a new problem, illustrated here:

\begin{minted}{java}
public class TestPoint {
    public static void main( String args[] ) {
        Point p1 = new Point();  // p1 should represent a point on the screen
        p1.x = -35;   // But can p1's x coordinate be negative?
        p1.y = -17;   // Or its y coordinate?
    }
}
\end{minted}

What happens here might surprise you.  We human readers of the English language know that a point on the screen isn't supposed to have negative values, but the Java compiler doesn't understand English.  Java \textit{will not} interpret this as an error, because to Java it isn't an error.  It's a \textit{semantic error}, because we know it's wrong, but it doesn't violate the rules of the Java language, so it will run and be wrong.

\begin{defn}{Semantic Error}
    A \textbf{semantic error}\index{Error!Semantic error} is an error in logic or meaning.  Typically, we notice semantic errors in code that has compiled and is running, because only a program that's running can generate incorrect output.
\end{defn}

\section{Visibility}

Thus, we learn the most important responsibility of Java programmers: We must protect the data in our classes, because relying on others to do the right thing will not work.

\begin{tip}{Data Protection}
    When designing a Java class, you must ensure that the data in its instance variables will always be valid.  If you don't work to ensure this, then there will come a point where the data will become invalid.
\end{tip}

The path to ensuring data validity starts with changing the \textit{visibility} of the instance variables.  We will mark \texttt{x} and \texttt{y} as \mintinline{java}{private}, like this:

\begin{minted}{java}
public class Point {
    private int x;  // x component of Cartesian ordered pair
    private int y;  // y component of Cartesian ordered pair
}
\end{minted}

The value of a \mintinline{java}{private} variable can not be retrieved or changed by any code outside the class.  This protects our data from being made negative by the \texttt{TestPoint} class:

\begin{minted}{java}
public class TestPoint {
    public static void main( String args[] ) {
        Point p1 = new Point();  // p1 should represent a point on the screen
        p1.x = -35;   // This is now a syntax error!
        p1.y = -17;   // And so is this!  Success!
    }
}
\end{minted}

But we have now done too good a job of protecting our data, because we can't set the variables with good data.  We can't even print their values!

\begin{minted}{java}
public class TestPoint {
    public static void main( String args[] ) {
        Point p1 = new Point();  // p1 represents a point on the screen
        p1.x = 10;   // This seemingly-valid request is now a syntax error
        p1.y = 15;   // So is this

        // And we can't even do this any more:
        System.out.println( "The point is at (" + p1.x + "x" + p1.y + ")." );
    }
}
\end{minted}

The instance variables still need to be \mintinline{java}{private}; the way to fix this problem is to give limited \mintinline{java}{public} access to viewing and changing the instance variables' values.  We will explore how to grant this access in the next chapter.
