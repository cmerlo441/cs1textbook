% !TEX root = cs1textbook.tex

\chapter{Storing Information and Working with Numbers}
\label{chapter:info-numbers}

\minitoc

\section{Using the Computer's Memory}
\label{sec:using-memory}

Virtually every computer program has to store information in the computer's memory.  In Java, we instruct the computer to do this for us by creating a \textbf{variable.}\index{Variable}

\begin{defn}{Variable}
A \textbf{variable} is a particular spot in the computer's memory to which we assign a name.
\end{defn}

Different programming languages differ in the amount of information we are required to give when we want to create a variable.  Java requires us to specify a unique \textit{identifier} for every variable we create, and it also requires us to state what \textit{type} of data the variable will store.

We specify this information in a \textbf{declaration statement}\index{Statement!Declaration}.

\begin{javaformat}{Declaration Statement}
\begin{minted}{java}
<data-type> <identifier>;
\end{minted}
\end{javaformat}

Many data types are provided by the Java language; we will discuss two of them later in this chapter.  We can also create our own data types, and that will be discussed later in this textbook.

Java imposes several rules about identifiers:
\bi
\item An identifier \textbf{must} start with a letter.
\item An identifier \textbf{must} only contain letters, digits, and the underscore character (\lstinline{'_'}).
\item An identifier \textbf{must not} be the same as a \textit{reserved word} in Java (like \lstinline{class} or \lstinline{static}).
\ei

If you break any of those rules, your source code will not become a program.  In addition to those rules, we programmers impose stylistic rules on ourselves, to make our source code easy to read:

\bi
\item An identifier \textit{should} start with a lowercase letter, unless it's the name of a class or a symbolic constant.
\item An identifier \textit{should} say something about the data stored in the variable.
\item A multi-word identifier \textit{should} \lstinline{use_underscores} or \lstinline{useCamelCasing} so that we can more easily pick out the individual words.\footnote{This author has noticed that the use of underscores in identifiers has fallen out of favor, and that more and more programmers use camel casing more and more of the time.}
\ei

While breaking those last three rules won't prevent your code from becoming a program, it makes your source code much harder to read, which makes it harder to fix and maintain in the future.\footnote{Note that this may also make it harder to earn good grades at school or positive reviews at work.}

Finally, when choosing a name for a variable, since a variable is a \textit{thing}, the name you choose should be a \textit{noun}.

\section{The Integers}
\label{sec:integers}

An \textbf{integer}\index{Integer}, according to \href{https://en.wiktionary.org/wiki/integer}{Wiktionary}, is a ``number that is not a fraction.''  These numbers are useful in everyday life when counting things that are not, or should not be, divisible, like the number of people you've invited to a party.

In Java, the name of the data type that we use to store integers is \mintinline{java}{int}\index{Primitive Data Types!Int@\mintinline{java}{int}}.  Consider this declaration statement:

\begin{minted}{java}
int x;
\end{minted}

This statement says to the CPU ``I would like to create a variable called \texttt{x} that stores an integer.''

\subsection{Internal Documentation}

Does the name \texttt{x} convey any meaning to you?  If you read only that line of code, would you know \textit{why} the programmer wrote it, or what is meant to be stored there?

We can make that declaration statement much more readable, and therefore more helpful to our future selves as well as to our future collaborators, by doing two things.  First, let's add a \textbf{comment} to the declaration.

\begin{defn}{Comment}
A \textbf{comment} is a section of source code that is \textit{completely ignored} by the computer, but readable to any human who reads the source code.  Comments are typically used for reminders and explanations.
\end{defn}

Consider this declaration:

\begin{minted}{java}
int x;  // The number of guests we invited
\end{minted}

That comment (which starts at \lstinline{//} and ends at the end of the line) helps the readability of this declaration.  However, it might still be difficult later in the code to remember what \lstinline{x} was supposed to store.  This is why programmers should\footnote{But not all programmers do, sadly!} choose \textit{meaningful} identifiers, as another way of internally documenting the program.

\begin{minted}{java}
int numberOfGuests;  // The number of guests we invited
\end{minted}

Now you can glance at only that line of code and be reasonably confident you know what it does.  More importantly, when you see that variable name later in the program, you'll remember why you created it.

\section{Assignment Statements}

The declaration statement we wrote in the previous section is very nice looking, and the variable it created is ready to store our integer data.  In order to make a variable store a piece of data, we must learn how to write an \textbf{assignment statement}\index{Statement!Assignment}.

\begin{javaformat}{Assignment Statement}
\begin{minted}{java}
<variable-name> = <value>;
\end{minted}
\end{javaformat}

As you can see, we type the name of the variable whose value we wish to change on the left hand side of an equal sign, and then we place the value we're trying to store on the right hand side of the equal sign.  Some programmers are fond of using the terms \textit{lvalue} and \textit{rvalue} to categorize the kind of things we are allowed to type on either side of the equal sign.

If we have invited fifteen people to a party, we may want to store the number 15 in the variable we declared before.  Here's that declaration again, along with an assignment statement that stores 15:

\begin{minted}{java}
int numberOfGuests;  // The number of guests we invited

numberOfGuests = 15;
\end{minted}

Now we can say that the variable \lstinline{numberOfGuests} stores the value 15.

Once you've declared a variable, and you want to use it, just type the identifier; \textit{do not} include the data type again.  The computer will think you're trying to create your variable all over again, which is not allowed.  In short, don't do this:

\begin{trap}{Re-Declaration}
\begin{minted}{java}
int numberOfGuests;  // The number of guests we invited

int numberOfGuests = 15;  // Oops!
\end{minted}
\end{trap}

\subsection{Initialization Statement}

We are allowed to give a variable an initial value when we declare it.  Some programmers call such a statement an \textbf{initialization statement}\index{Statement!Initialization}.  We can replace the three lines of code above with this:

\begin{minted}{java}
int numberOfGuests = 15;  // The number of guests we invited
\end{minted}

\subsection{Variables are Single-Valued}

A variable such as \lstinline{numberOfGuests} can only store one piece of data; when you write another assignment statement, only the newest value is retained, and any old data is discarded.

\begin{minted}{java}
int numberOfGuests = 15;

numberOfGuests = 16;  // The 15 is gone forever, and only the 16 is stored
\end{minted}

\begin{magic}{Output}
One of the simplest ways to generate program output is by sending a message to \textit{standard output} (we describe what that means in Section \ref{sec:console-io}).  We do this by \textit{calling a method} called \lstinline{println()} (pronounced ``print line'') on the standard output \textit{object}.  To display the value of the variable \lstinline{x}, we would write:\\

\mintinline{java}{System.out.println( x );}
\end{magic}

You can prove to yourself that variables are single-valued by displaying the value of \mintinline{java}{numberOfGuests} after changing it:

\label{code:variables}
%\javafile{code/Variables.java}

\section{The Real Numbers}
\label{sec:double}

Not all situations warrant the use of integers to store numeric data.  You may, for instance, need to store the weight of the fish you caught last weekend:

\begin{minted}{java}
int weight = 8.6;  // Uh oh!
\end{minted}

The initialization above breaks the rules of the Java language, because an integer can't store a number with a decimal point.  This kind of error is called a \textit{type mismatch}.

\begin{defn}{Type Mismatch}\label{defn:typemismatch}
A \textbf{type mismatch} error occurs when the \textit{lvalue} and \textit{rvalue} in an assignment statement are different data types, and the \textit{rvalue} can not automatically be converted to the \textit{lvalue} type.
\end{defn}

This is where \textit{floating-point numbers}, including the \mintinline{java}{double}\index{Primitive Data Types!Double@\mintinline{java}{double}} data type, come in.

\begin{defn}{Floating-Point Number}
A \textbf{floating-point number}\index{Floating-Point Number} is a number that has a fractional component.
\end{defn}

Examples of floating-point numbers are $21.12$, $-5.150$, and $3.14159$.  Notice that $6.0$ is also a floating-point number, even though the fractional component is zero.  Java provides two built-in data types that we can use to store floating-point numbers: \mintinline{java}{float}, which is short for ``single-precision floating-point number'', and \mintinline{java}{double}, which is short for ``double-precision floating-point number''.  As the name suggests, the \mintinline{java}{double} data type stores twice as much data about a number as \mintinline{java}{float} does, and so most Java programmers choose \mintinline{java}{double} when the need to store a floating-point number arises.

Now that we know the proper data type to use, we can succesfully declare and initialize a variable to store the weight of our fish:

\begin{minted}{java}
double weight = 8.6;  // Weight of the fish we caught
\end{minted}

Notice that a floating-point number stores \textit{more} information than an integer does, and so the computer can generally convert an \mintinline{java}{int} to a \mintinline{java}{double} without any trouble.  This statement will automatically \textit{promote} the integer to a floating-point number:

\begin{minted}{java}
double x = 5;  // Automatic promotion; not a type mismatch!
\end{minted}

\section{Arithmetic}

The \textit{rvalue} of an assignment statement doesn't have to just be a number.  We can perform arithmetic as well.  Consider this example:

\begin{minted}{java}
int x = 5;
int y = 11;
int z = x + y;  // z stores 16
\end{minted}

Table~\ref{table:operators} on page~\pageref{table:operators} shows the arithmetic operators you can use in Java.

\begin{table}[h]
\begin{tcolorbox}[sharp corners=downhill,tabularx={c|c}]
\textbf{Operator} & \textbf{Meaning}\\
\hline
\texttt{+} & Addition\\
\texttt{-} & Subtraction\\
\texttt{*} & Multiplication\\
\texttt{/} & Division\\
\texttt{\%} & Modulo\\
\hline
\end{tcolorbox}
\caption{Mathematical Operators in Java}
\label{table:operators}
\end{table}

% \begin{ttable}{Mathematical Operators in Java}
%     \begin{tabular}{c|c}
%     \textbf{Operator} & \textbf{Meaning}\\
%     \hline
%     \texttt{+} & Addition\\
%     \texttt{-} & Subtraction\\
%     \texttt{*} & Multiplication\\
%     \texttt{/} & Division\\
%     \texttt{\%} & Modulo\\
%     \hline
%     \end{tabular}
% \end{ttable}

You can see the effects of some of these operations in this program:

\begin{listing}[H]
\caption{Determine the Longer String}
\inputminted{java}{code/Arithmetic.java}
\label{code:arithmetic}
\end{listing}

The value that gets assigned to \mintinline{java}{d1} in this example might surprise you.  In math class, $6 \div 4$ should give an answer of $1.5$.  However, in the Java language, whenever you divide an integer by an integer, the answer will always be an integer, and any decimal portion of the answer will be \textit{truncated}, so that \mintinline{java}{d1} receives the number 1, not the number 1.5.

In addition, it's important to know that, as Java programers say, \textit{``Assignment works from right to left''}, which means that the rvalue is completely evaluated before any assignment takes place, so the number 1 is assigned to \lstinline{d1} (which then gets promoted to $1.0$).

There are a couple of ways to make \mintinline{java}{d1} store 1.5 in this example, all of which rely on understanding that all the integers in a mixed-mode expression will be \textit{promoted} to doubles, which means that they will temporarily be treated like doubles so that none of the floating-point data will be lost.

First, we could change the integer 4 on line 11 to 4.0:

% Had to remove firstnumber=11 here
\begin{minted}[firstnumber=11]{java}
        d1 = i1 / 4.0;
\end{minted}

This will work, but only in this case; you will frequently come across situations where neither operand is a literal number, and so appending a ``.0'' on the end won't be feasible.

You could also multiply one of the operands by the double 1.0, promoting it to a double before the multiplication takes place:

% and here
\begin{minted}[firstnumber=11]{java}
        d1 = ( i1 * 1.0 ) / 4;
\end{minted}

What many programmers do is \textit{cast} one of the integers to double.

\begin{defn}{Casting}
\textbf{Casting}\index{Casting} a value means temporarily pretending it's a value of another type.  The cast is not permanent; once the statement ends, the value is treated like its original type again.
\end{defn}

We cast a value by placing the target data type in parentheses before the value to be cast.  This code casts \lstinline{i1} to \lstinline{double} before dividing it by 4:

\begin{minted}[firstnumber=11]{java}
        d1 = ( double ) i1 / 4;
\end{minted}

All three of the examples above will generate the value 1.5 when dividing, and therefore store the value 1.5 in \lstinline{d1}.  Remember that the order of operations applies, so don't misuse the parentheses like this:


\begin{trap}{Casting}
\begin{minted}{java}
d1 = ( double )( i1 / 4 );  // Uh oh!  This stores 1 in d1!
\end{minted}
\end{trap}

\subsection{Is That a Bug Or a Feature?}

It is certainly possible to have a mathematical mistake in your code if you are not aware of the rules about division.  But is that a bad thing?  If you know how the computer will respond to the code you write, then there should be no surprises; therefore, dividing 6 by 4 and getting an answer of 1 isn't a mistake if you were expecting it.

It is also important to note that this property of the integers can be harnessed for problem solving.  Recall that we created the variable \lstinline{numberOfGuests} for planning a party.  If we know how many people can fit at one table, we can use the rules of the integers to our advantage:

\begin{minted}{java}
int numberOfGuests = 15;
int guestsPerTable = 4;
int numberOfTables = ( numberOfGuests / guestsPerTable ) + 1;
\end{minted}

\subsection{So Why Are There Two Numeric Types?}

Due to the limitations inherent in computer hardware, floating-point numbers are sometimes approximated.  There are only so many digits of an irrational number that can be stored in memory, and very weird things can happen when we run out of room.  If we calculate the square root of 2.0 and store it in a \mintinline{java}{double} variable called \mintinline{java}{d}, and then multiply \mintinline{java}{d} by itself, the value of \mintinline{java}{d} will wind up being $2.0000000000000004$.

Remember that a \mintinline{java}{double} \textit{always} potentially stores an approximation, and so you should always use \mintinline{java}{int} when storing an integer.
