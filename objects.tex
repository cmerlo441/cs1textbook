% !TEX root = /Users/cmerlo/cs1textbook/cs1textbook.tex

\chapter{Objects and Classes}

\minitoc

(taken out of chapter 4)

There are some important differences to be noted between the declaration and initialization of primitive variables and the variable \lstinline{s1}.  First, notice that the data type \lstinline{String} starts with a capital letter, whereas the names of the primitive data types all start with lowercase letters.  This initial capital letter signifies that \lstinline{String} is the name of a \textit{class}.

Because \lstinline{String} is a class, the variable \lstinline{s1} is a special kind of variable called a \lstinline{reference variable}.  We will come back to this topic in a moment.

Also notice that the creation of a String variable requires\footnote{As far as you know.} the use of the keyword \lstinline{new}, followed by the data type \lstinline{String} again and then some parentheses.  Since \lstinline{String} is a class, variables of this type must be \textit{instantiated}.

When this instantiation -- the creation of an \textit{instance} of the \lstinline{String} class -- occurs, rather than storing the instance directly in the variable \lstinline{s1}, what \lstinline{s1} stores is a \textit{reference} to the instance.  This reference is an address in memory.

\subsection{Memory, Primitives, and References}
A computer's memory is made up of millions of tiny devices, each of which is capable of storing one piece of binary information -- one binary digit, or one \textit{bit}.  Each such device can be in one of two states, typically referred to as 0 or 1.  A collection of eight such devices can store eight 0s and 1s; such a group is called a \textit{byte}.

A typical computer operating system will give each byte in memory a numeric address, starting at 0 and increasing sequentially through the memory.
