% !TEX root = cs1textbook.tex

\chapter{Input and Output}

\minitoc

In this chapter, we will explore two different systems that we can exploit for input and output; that is, allowing our program's user to give information to our programs, and allowing our programs to send information out to the user.

\section{Console Input and Output}
\label{sec:console-io}
Recall that the word \textit{console}, as defined in Subsection \ref{subsection:console}, can be used to refer to hardware devices, but also to a method of performing input and output.  Graphical user interfaces will be introduced in the next section (Section \ref{section:gui}), but this section will focus on a more ``old fashioned'' method of getting data in and out of the program.

\subsection{Data Streams}
Every operating system provides every program with a low-level way for collecting data from the operating system, and two low-level ways for sending data back out.  These are called \textit{streams}, as illustrated in this table:

\begin{table}[h]
\begin{tcolorbox}[tabularx={X|p{4.5in}}]
\textbf{Stream} & \textbf{Description}\\
\hline
Standard Input & A buffered stream for keyboard input\\
Standard Output & A buffered stream for monitor output; used for normal program output\\
Standard Error & An \textit{unbuffered} stream for monitor output; used for exceptional program output, like error messages\\
\hline
\end{tcolorbox}
\caption{Input/Output Streams}
\label{table:streams}
\end{table}



\section{Graphical User Interfaces}
\label{section:gui}
Foo.
