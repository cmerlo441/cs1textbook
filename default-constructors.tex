% !TEX root = cs1textbook.tex

\chapter{Default Constructors and Accessors}
\label{chapter:default-constructors}

\minitoc

At the end of Chapter \ref{chapter:classes} we decided that instance variables should be \mintinline{java}{private}, and that we should carefully grant \mintinline{java}{public} access as necessary so that the instance variables can be set to store appropriate values.  In this chapter, we discuss how to store appropriate values at the time an object is created.

\section{Constructors}

The reason that the code \mintinline{java}{Point p1 = new Point();} looks like a method call is because it \textit{is} a method call.  When we use \mintinline{java}{new} in this way to instantate an object, the Java interpreter calls a method called a \textit{constructor}.

\begin{defn}{Constructor}
    A \textbf{constructor}\index{Methods!Constructors} is a special method in a class that runs automatically when an instance of that class is created.
\end{defn}

There are two very important things to know about constructors, that violate previous rules and guidelines about methods in general:
\bi
\item The name of the constructor method must be exactly the same as the name of the class.
\item Constructor methods do not have a return type; not even \mintinline{java}{void}.
\ei

When you write a class, it is your responsibility to ensure that the data stored in the instance variables are always valid.  This responsibility begins at the moment an object of the class is created.  Therefore, every constructor we write must ensure that every instance variable is initialized with valid data values.

\begin{tip}{Constructors}
    Make sure that every constructor you ever write initializes all the class' instance variables with valid data values.
\end{tip}

\subsection{Default Constructors}

There are two basic kinds of constructors.  We will be discussing \textit{parameterized constructors} in Section $X$, but this discussion will focus on \textit{default constructors}.  A \textbf{default constructor}\index{Methods!Constructors!Default constructors} is simply a constructor with an empty parameter list.  When you visit a new pizza place for the first time, and you aren't aware of what exotic toppings or crust styles they may have available, you might choose the default option -- ``I'll take one regular slice, please.''\footnote{This is virually always the right choice where I live; your mileage may vary, especially as you travel farther from New York City.}  When another programmer wants to instantiate your class, but isn't aware of what options exist, he or she will likely choose the default option as well, by calling your default constructor.

And so, when designing a new class, one of your responsibilities is to determine what will serve as good default values for your class.  The pizza chef will probably choose a white flour crust, tomato sauce, and mozzarella cheese.  For the point class, we might choose the one pixel we can guarantee exists -- the top left corner, $(0,0)$.

With all those details in place, we can finally write the \texttt{Point} class' default constructor:

\begin{minted}{java}
public class Point {
    /* ... */

    /**
     * Default constructor
     * <p>
     * Creates Point that represents (0,0)
     */
    public Point() {
        x = 0;
        y = 0;
    }
}
\end{minted}

\section{Accessors}

Now that the default constructor has been written, we might assume that code like:

\begin{minted}{java}
Point p1 = new Point();
\end{minted}

generates a \texttt{Point} class object that stores the coordinate $(0,0)$.  But we don't know this for sure until we can get the data back out of the object.  Remember that code like \mintinline{java}{System.out.println( p1.x );} will still result in a syntax error, because the instance variables are \mintinline{java}{private}.  Writing \textit{accessor methods} will solve this problem.

\begin{defn}{Accessor}
    An \textbf{accessor method}\index{Methods!Accessors} is a method that grants public access to private data.
\end{defn}

The generally accepted conventions for accessor methods are as follows:
\bi
\item Write one accessor per instance variable (or fewer -- you might not want to grant access to every instance variable)
\item Begin the method's identifier with \texttt{get}, or \texttt{is} if the instance variable is of type \mintinline{java}{boolean}
\item Finish the method's identifier with the name of the instance variable, capitalizing the first letter
\item The return type of the method should be the same as the type of the instance variable
\item The method should not do anything other than return the value of the instance variable
\ei

Following these conventions, the accessor method for the \texttt{x} instance variable should look like this:

\begin{minted}{java}
    /**
     * Returns x value of ordered pair
     * @return Point's x value
     */
    public int getX() {
        return x;
    }
\end{minted}

And \texttt{y}'s accessor should look like this:

\begin{minted}{java}
/**
 * Returns y value of ordered pair
 * @return Point's y value
 */
    public int getY() {
        return y;
    }
\end{minted}

With that, the \texttt{Point} class now looks like this:

\begin{minted}{java}
/**
 * Provides a two-dimensional point on a Cartesian plane.
 * <p>
 * All x and y values are non-negative.
 */
 public class Point {
    private int x;  // x component of Cartesian ordered pair
    private int y;  // y component of Cartesian ordered pair

    /**
     * Default constructor
     * <p>
     * Creates Point that represents (0,0)
     */
    public Point() {
        x = 0;
        y = 0;
    }

    /**
     * Returns x value of ordered pair
     * @return Point's x value
     */
    public int getX() {
        return x;
    }

    /**
     * Returns y value of ordered pair
     * @return Point's y value
     */
    public int getY() {
        return y;
    }
}
\end{minted}

\section{Testing}

Now that the accessor methods exist for the \texttt{Point} class, we can use them to test the default constructor.  In general, you should test your code frequently -- every time you add or edit a new method, you should run it a few times, and run all the other code that depends on this new code, to make sure all the interactions are working the way you want.  One minor change can have unintended consequences in some other part of your code, but if you test frequently, you can greatly reduce the amount of code you have to search through to find your bug.

\begin{tip}{Run Your Code Early and Often}
    something about bugs snowballing
\end{tip}

Now we can write an application like this to make sure that our default constructor has the effect we want:

\begin{minted}{java}
public class TestPoint {
    Point p1 = new Point();  // we assume this Point contains (0,0)

    System.out.println( "p1's x value is " + p1.getX() );
    System.out.println( "p1's y value is " + p1.getY() );
}
\end{minted}

When we compile and run these two classes together, we get the output we expect, confirming that the default constructor does what we want it to.

\begin{minted}{shell-session}
$ javac Point.java TestPoint.java
$ java TestPoint
p1's x value is 0
p1's y value is 0
\end{minted}
